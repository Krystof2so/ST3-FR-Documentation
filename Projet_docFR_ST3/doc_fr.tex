\documentclass[french,a4paper]{article}

\usepackage[T1]{fontenc}
\usepackage[utf8]{inputenc}
\usepackage[french]{babel}
\usepackage{color}

\usepackage{hyperref}

\title{SublimeText 3}
\author{Krystof2so}
\date{\today}

\begin{document}
\maketitle

\section{\textcolor{magenta}{Juste pour se familiariser}}
Pour tout ce qui concerne l'utilisation de \texttt{SublimeText} l'utisateur
 est invité à se rendre dans le répertoire:
\begin{verbatim}
    ~/.config/sublime-text-3
\end{verbatim}
\medskip

Le répertoire \verb|/Packages| contient le répertoire \verb|/User| qui est le
 répertoire fourre-tout de l'utilisateur ((accessible par \texttt{Preferences}
 -> \texttt{Browse Packages}). On y retrouve les greffons personnalisés, les
 \textit{snippets}, les macros, etc. Il est à considérer comme votre espace 
 personnel qu'aucun mise à jour de \texttt{SublimeText} n'écrasera.
\medskip

Il est à noter que \texttt{SublimeText} est programmable, ce qui permet aux
utilisateurs ayant des compétences en programmation d'ajouter leurs propres
fonctionnalités, en utilisant le langage \texttt{Python}.
\medskip

\texttt{SublimeText} est doté d'une console \texttt{Python} que l'on peut
 ouvrir via \texttt{View} -> \texttt{Show Console}. Ainsi, un interpréteur
 \texttt{Python} est intégré à l'éditeur (séparé de l'interpréteur 
 \texttt{Python} du système. Cet interpréteur est alors utile pour inspecter
 les paramètres de l'éditeur et pour tester rapidement des appels d'API lors du
 développement de greffons. 
\fcolorbox{red}{yellow}{Capture d'écran}.
\medskip

Presque tous les aspects du \texttt{SublimeText} peuvent être étendus ou
 personnalisés. La grande flexibilité de \texttt{SublimeText} explique pourquoi
 il y a tant de fichiers de configuration. Les fichiers de configuration de 
 \texttt{SublimeText} sont des fichiers texte qui se conforment à une structure
 ou un format prédéfini : \texttt{JSON} prédomine, mais on trouve aussi des 
 fichiers \texttt{XML} et \texttt{YAML}. Pour les options d'extensibilité plus
 avancées ce sont des fichiers de code source \texttt{Python} qui sont 
 utilisés.
\medskip

D'autre part \texttt{SublimeText} fournit une émulation de \texttt{vim} grâce 
 au paquet \texttt{Vintage}. Ce paquet est ignoré par défaut et doit être 
 activé par l'utilisateur\footnote{Voir à ce sujet: 
 \url{https://www.sublimetext.com/docs/3/vintage.html}}.
\bigskip


\section{\textcolor{magenta}{Pour une utilisation basique}}

\subsection*{L'édition}
Les fonctions d'édition sont multiples et variées.
\bigskip

\subsection*{La gestion des fichiers et des projets}
Les projets regroupent des ensembles de fichiers et de dossiers qui vous
 permettent de garder votre travail organisé. 
\medskip

Il y a toujours un projet actif, même si vous n'en avez pas créé ou ouvert un.
 Dans cette situation, vous travaillez avec un projet anonyme, dont les
 fonctionnalités sont limitées. Les nouvelles fenêtres utilisent toujours un
 projet anonyme lorsqu'elles s'ouvrent pour la première fois.
\medskip

Les métadonnées du projet sont stockées dans des fichiers \texttt{JSON} avec 
 une extension \texttt{.sublime-project}. Partout où il y a un fichier 
 \texttt{.sublime-project}, vous trouverez également un fichier 
 \texttt{.sublime-workspace} auxiliaire. Ce fichier contient des données de 
 session que vous ne devez jamais modifier. 
\medskip

\textbf{Créer un projet}: \\
Commencer par un projet anonyme en ouvrant une nouvelle fenêtre ou en fermant
 tout projet actif avec le menu \texttt{Project} -> \texttt{Close project}.
 Nous pouvons alors ajouter et supprimer des dossiers à un projet à l'aide du 
 menu \texttt{Project} ou du menu contextuel de la barre latérale. Faire 
 glisser un dossier sur une fenêtre de \texttt{SublimeText} ajoutera également
 le fichier au projet. Pour enregistrer un projet anonyme, se rendre sur dans
 \texttt{Project} -> \texttt{Save Project As\dots}. Une fois le projet 
 enregistré, il demeure modifiable.
 \medskip

\textbf{Ouvrir des projets}: \\ 
L'onglet \texttt{Project} fournit diverses
 possibilité d'ouverture. De plus, vous pouvez ouvrir un projet en ligne de
 commande en passant le fichier \texttt{.sublime-project} en argument à l'aide
 de la commande permettant l'ouverture de Sublime Text.
\medskip

\textbf{Les espaces de travail (\textit{Workspaces})}: \\ 
Les espaces de travail contiennent des données de session associées à un 
 projet, qui comprennent des informations sur les fichiers ouverts, la 
 disposition des volets, l'historique des recherches et bien plus encore. Un 
 projet peut avoir plusieurs espaces de travail. Les espaces de travail se 
 comportent comme des projets. Pour créer un nouvel espace de travail, 
 sélectionnez \texttt{Project} -> \texttt{New Workspace For Project}. Pour
 enregistrer l'espace de travail actif, sélectionnez \texttt{Project} -> 
 \texttt{Save Project As...}. Les métadonnées de l'espace de travail sont 
 stockées dans des fichiers \texttt{JSON} et portent l'extension 
 \texttt{.sublime-workspace}, que vous n'êtes pas censé modifier. Contrairement
 aux fichiers \texttt{.sublime-projet}, les fichiers 
 \texttt{.sublime-workspace} ne sont pas destinés à être partagés, vous ne 
 devez donc jamais livrer des fichiers \texttt{.sublime-workspace} dans un 
 dépôt de code source.
\bigskip

\subsection*{La navigation}
\textbf{Goto Anything}: \texttt{[Ctrl] + T} \\
En utilisant \texttt{Goto Anything}, vous pouvez naviguer rapidement dans les
 fichiers de votre projet. \texttt{Goto Anything} accepte divers opérateurs. \\
\medskip

Le symbole \og \texttt{:}\fg{} : \texttt{[Ctrl] + G}
\begin{verbatim}
    fichier1:100
\end{verbatim}

Cela indique à Sublime Text de rechercher d'abord un fichier dont le chemin
 d'accès correspond à \textit{fichier1}, puis de se rendre à la ligne 100 dans
 le dit fichier. \\
\medskip

Le symbole \texttt{\@}: \texttt{[Ctrl] + R}
\begin{verbatim}
    @ma_variable
\end{verbatim}

Recherche dans le fichier actif le symbole nommé \textit{ma\_variable}. Les
symboles comprennent généralement des noms de classe et de fonction.
\medskip

Le symbole \texttt{\#}: \texttt{[Ctrl] + ;}
\begin{verbatim}
    #verb
\end{verbatim}

Recherche les termes dans lequel va figurer l'exression \textit{verb}.
\medskip

La barre latérale donne un aperçu du projet actif. Les fichiers et dossiers de
 la barre latérale seront disponibles dans le menu \texttt{Goto Anything}. 
 Projets et barre latérale sont étroitement liés. Il est important de noter 
 qu'il y a toujours un projet actif ; si vous n'avez pas ouvert de fichier de 
 projet, un projet anonyme sera utilisé à la place. La barre latérale fournit 
 des opérations de base de gestion de fichiers grâce à son menu contextuel. 
\bigskip

\subsection*{Rechercher et remplacer}
\textbf{Sur fichier unique}: \\
\begin{description}
	\item[Ouvrir le panneau de recherche]: \texttt{[Ctrl] + F}
	\item[Ouvrir le panneau de recherche progressive]: \texttt{[Ctrl] + I}
	\item[Basculer sur le mode des expressions régulières]: \texttt{[Alt] + R}
	\item[Sensibilité à la casse]: \texttt{[Alt] + C}
	\item[Correspondance exacte]: \texttt{[Alt] + W}
	\item[Ouvrir le panneau de remplacement]: \texttt{[Ctrl] + H}
	\item[Tout remplacer]: \texttt{[Ctrl] + [Alt] + [Entrée]}
\end{description}
\bigskip

\textbf{Sur plusieurs fichiers}:\\
\begin{description}
	\item[Ouvrir le panneau de recherche]: \texttt{[Ctrl] + [Shift] + F} 
	\item[Basculer sur le mode des expressions régulières]: \texttt{[Alt] + R}
	\item[Sensibilité à la casse]: \texttt{[Alt] + C}
	\item[Correspondance exacte]: \texttt{[Alt] + W}
\end{description}
\medskip

Il est également possible de paramétrer la portée de la recherche, en
 définisant les champs de recherche par l'ajout de répertoires individuels
 (chemins d'accès de type Unix, même sous Windows), par l'ajout/exclusion de
 fichiers basés sur des jokers (\og *\fg{}) ou l'ajout d'emplacements 
 symboliques (\texttt{<open folders>}, \texttt{<open files>} \dots). Il est 
 également possible de combiner ces filtres en utilisant des virgules.
\medskip

Vous pouvez naviguer dans les résultats de recherche à l'aide des raccourcis
suivants:
\begin{description}
	\item[Occurence suivante]: \texttt{[F4]}
	\item[Occurence précédente]: \texttt{[Shift] + [F4]}
	\item[Se rendre d'une occurence à une autre]: \texttt{[Entrée]}
\end{description}
\bigskip

\textbf{Les expressions régulières (\textit{regexp} ou \textit{regex})}:\\
Les expressions régulières permettent de trouver des motifs complexes dans un
 texte. Pour profiter pleinement des possibilités de recherche et de 
 remplacement, vous devriez au moins apprendre les bases des expressions 
 régulières. Pour les utiliser vous devez d'abord les activer dans les 
 différents panneaux de recherche (\texttt{[Alt] + R}), sinon, les termes 
 recherchés seront interprétés littéralement.
\bigskip

\section{\textcolor{magenta}{Personnalisation de \texttt{Sublime Text}}}
Voyons les options les plus courantes afin de personnaliser entièrement Sublime
 Text.
\bigskip

\subsection*{Les paramètres (\textit{Settings})}
\texttt{Sublime Text} stocke les données de configuration dans des fichiers 
 portant l'extension \texttt{.sublime-settings}. La flexibilité s'obtint au
 prix d'un système légèrement complexe s'agissant des paramètres. Il faudra
 veillez à toujours placer ses fichiers de configuration personnels, afin de
 garantir qu'ils auront la priorité sur tout autre fichier de configuration,
 dans le répertoire:
\begin{verbatim}
    ~/.config/sublime-text-3/Packages/User
\end{verbatim}
\medskip

Ces fichiers de paramètrage sont au format \texttt{JSON}. Le nom de chaque 
 fichier de type \texttt{.sublime-settings} détermine sa finalité. Les noms 
 peuvent être descriptifs (comme \texttt{Preferences.sublime-settings} ou 
 \texttt{Minimap.sublime-settings}, ou ils peuvent être liés aux paramètres de
 contrôle. Les fichiers syntaxiques propres aux langages de programmation sont
 aussi reconnaissables (exemple: \texttt{Python.sublime-settings}). En outre, 
 certains fichiers de paramètres ne s'appliquent qu'à des plates-formes 
 spécifiques. Cela peut se déduire à partir du nom du fichier, comme par 
 exemple \texttt{Preferences (\{Linux\}).sublime-settings}. A noter que de tels
 fichiers de réglages spécifiques à une plate-forme situés dans le répertoire
 \texttt{Packages/User} sont ignorés. De cette façon, vous pouvez être sûr 
 qu'un seul fichier de ce type a priorité sur tous les autres. Enfin, sachez
 que les modifications de paramètres sont généralement prises en compte par 
 \texttt{Sublime Text} en temps réel, mais vous devrez peut-être redémarrer 
 l'éditeur afin de charger de nouveaux fichiers de paramètres.
\medskip

Pour accéder aux fichiers de paramètres et les modifier il faudra passer par le
 menu \texttt{Preferences} -> \texttt{Settings}. Seront alors ouverts en 
 parallèle les fichiers:
\begin{center}
\begin{verbatim}
    Preferences.sublime-settings User
\end{verbatim}
\medskip
et
\begin{verbatim}
    Preferences.sublime-settings Default.
\end{verbatim}
\end{center}
L'édition de ce dernier est fortement déconseillée, mais vous pouvez utiliser 
 ce fichier à titre de référence puisqu'il contient les commentaires expliquant
 la finalité de tous les paramétrages globaux disponibles.
\medskip

Le même fichier de paramètres (tel que Python.sublime-settings) peut
apparaître à plusieurs endroits. Tous les paramètres définis dans des fichiers
de nom identique seront fusionnés et écrasés selon des règles prédéfinies. 
Ci-dessous, vous pouvez voir l'ordre dans lequel \texttt{Sublime Text} 
traiterait une hiérarchie hypothétique de paramètres pour les fichiers Python:
\begin{itemize}
    \item Packages/Default/Preferences.sublime-settings
    \item Packages/Default/Preferences (votre\_système).sublime-settings
    \item Packages/User/Preferences.sublime-settings
    \item Packages/Python/Python.sublime-settings
    \item Packages/User/Python.sublime-settings 
\end{itemize}
\bigskip

\subsection*{Les raccourcis claviers (\textit{Key Bindings})}
Les combinaisons de touches sont définies au format \texttt{JSON} et stockées 
 dans des fichiers portant l'extension \texttt{.sublime-keymap}. Dans le même 
 répertoire, des fichiers \textit{keymap} séparés pour \texttt{Linux}, 
 \texttt{OSX} et \texttt{Windows} peuvent co-exister pour une meilleure 
 intégration au système d'exploitation. Exemple:
\begin{verbatim}
    [
        { "keys": ["ctrl+shift+n"], "command": "new_window" },
        { "keys": ["ctrl+o"], "command": "prompt_open_file" }
    ]
\end{verbatim}
\medskip

\texttt{Sublime Text} est livré avec des raccourcis clavier par défaut. Afin de
 pouvoir les remplacer ou d'en ajouter de nouveaux, utilisez un fichier 
 \textit{keymap} séparé avec une priorité plus élevée en le plaçant dans le 
 répertoire \texttt{Packages/User}.
\medskip

Les raccourcis clavier simples consistent en une séquence d'une ou plusieurs 
 touches associées à une commande. Cependant, il existe des syntaxes plus 
 complexes pour passer des arguments aux commandes et pour restreindre les 
 raccourcis clavier à des contextes spécifiques. Exemple:
\begin{verbatim}
    [
        { "keys": ["shift+enter"], "command": "insert", "args": 
            {"characters": "\n"} }
    ]    
\end{verbatim}
\medskip

Dans cet exemple, \og \textbackslash{n}\fg{} est passé à la commande 
 d'insertion à chaque fois que l'on presse les touches \texttt{[Shift]} + 
 \texttt{[Entrée]}.  
\medskip

Vous pouvez créer raccourcis composés de plusieurs combinaisons de raccourcis.
Exemple:
\begin{verbatim}
    { "keys": ["ctrl+k", "ctrl+v"], "command": 
              "paste_from_history" }
\end{verbatim}
\medskip

Ici, pour déclencher la commande \textit{paste\_from\_history}, vous devez 
 d'abord presser les touches \texttt{[Ctrl] + K}, puis relâcher la touche 
 \texttt{K}, et enfin appuyer sur la touche \texttt{V}. Cet exemple est une
 liaison de touches par défaut que vous pouvez donc essayer.
 \bigskip

\subsection*{Les menus}
Il est possible de modifier les menus de \texttt{Sublime Text}.

Cf. \url{https://docs.sublimetext.io/guide/customization/menus.html}
\bigskip

\subsection*{Les schémas de couleurs}

\texttt{Sublime Text} utilise des schémas de couleurs pour mettre en évidence
 le code source et pour définir les couleurs de certains éléments de la zone 
 d'édition : arrière-plan, avant-plan, gouttière, curseur, sélection... Ces 
 schémas de couleurs sont entièrement personnalisables.
 \medskip

 Modification du schéma des couleurs: \\
 \texttt{[Preferences] -> [Color Scheme\dots]}
\bigskip

\section{\textcolor{magenta}{Les extensions pour \texttt{Sublime Text}}}
\subsection*{La palette de commandes}
Ouverture: \texttt{[Ctrl] + [Shift] + P}. Puis sélectionnert la commande 
 voulue. \\
La palette de commandes filtre les entrées selon le contexte. Cela signifie
 qu'à chaque fois que vous l'ouvrirez, vous ne verrez pas toujours toutes les
 commandes définies dans chaque fichier \texttt{.sublime-commands.}
\medskip

La palette de commandes est alimentée par des entrées dans les fichiers du 
 type \texttt{.sublime-commands}. Par défaut, la palette de commandes comprend
 de nombreuses commandes utiles et permet un accès pratique aux paramètres 
 individuels ainsi qu'aux fichiers de paramètres. Les fichiers de commandes 
 utilisent des fichiers de type \texttt{JSON}.

 \subsection*{La complétion}
Il existe plusieurs façons d'étendre la complétion en fonction de la syntaxe
utilisée.


\end{document}